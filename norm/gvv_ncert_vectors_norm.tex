\let\negmedspace\undefined
\let\negthickspace\undefined
%\RequirePackage{amsmath}
\documentclass[journal,12pt,twocolumn]{IEEEtran}
%
% \usepackage{setspace}
 \usepackage{gensymb}
%\doublespacing
%\singlespacing
%\usepackage{silence}
%Disable all warnings issued by latex starting with "You have..."
%\usepackage{graphicx}
\usepackage{amssymb}
%\usepackage{relsize}
\usepackage[cmex10]{amsmath}
%\usepackage{amsthm}
%\interdisplaylinepenalty=2500
%\savesymbol{iint}
%\usepackage{txfonts}
%\restoresymbol{TXF}{iint}
%\usepackage{wasysym}
\usepackage{amsthm}
%\usepackage{iithtlc}
% \usepackage{mathrsfs}
% \usepackage{txfonts}
% \usepackage{stfloats}
% \usepackage{steinmetz}
% \usepackage{bm}
% \usepackage{cite}
% \usepackage{cases}
% \usepackage{subfig}
%\usepackage{xtab}
\usepackage{longtable}
%\usepackage{multirow}
%\usepackage{algorithm}
%\usepackage{algpseudocode}
\usepackage{enumitem}
 \usepackage{mathtools}
% \usepackage{tikz}
% \usepackage{circuitikz}
% \usepackage{verbatim}
%\usepackage{tfrupee}
\usepackage[breaklinks=true]{hyperref}
%\usepackage{stmaryrd}
%\usepackage{tkz-euclide} % loads  TikZ and tkz-base
%\usetkzobj{all}
\usepackage{listings}
    \usepackage{color}                                            %%
    \usepackage{array}                                            %%
    \usepackage{longtable}                                        %%
    \usepackage{calc}                                             %%
    \usepackage{multirow}                                         %%
    \usepackage{hhline}                                           %%
    \usepackage{ifthen}                                           %%
  %optionally (for landscape tables embedded in another document): %%
    \usepackage{lscape}     
% \usepackage{multicol}
% \usepackage{chngcntr}
%\usepackage{enumerate}

%\usepackage{wasysym}
%\newcounter{MYtempeqncnt}
\DeclareMathOperator*{\Res}{Res}
%\renewcommand{\baselinestretch}{2}
\renewcommand\thesection{\arabic{section}}
\renewcommand\thesubsection{\thesection.\arabic{subsection}}
\renewcommand\thesubsubsection{\thesubsection.\arabic{subsubsection}}

\renewcommand\thesectiondis{\arabic{section}}
\renewcommand\thesubsectiondis{\thesectiondis.\arabic{subsection}}
\renewcommand\thesubsubsectiondis{\thesubsectiondis.\arabic{subsubsection}}

% correct bad hyphenation here
\hyphenation{op-tical net-works semi-conduc-tor}
\def\inputGnumericTable{}                                 %%

\lstset{
%language=C,
frame=single, 
breaklines=true,
columns=fullflexible
}
%\lstset{
%language=tex,
%frame=single, 
%breaklines=true
%}

\begin{document}
%


\newtheorem{theorem}{Theorem}[section]
\newtheorem{problem}{Problem}
\newtheorem{proposition}{Proposition}[section]
\newtheorem{lemma}{Lemma}[section]
\newtheorem{corollary}[theorem]{Corollary}
\newtheorem{example}{Example}[section]
\newtheorem{definition}[problem]{Definition}
%\newtheorem{thm}{Theorem}[section] 
%\newtheorem{defn}[thm]{Definition}
%\newtheorem{algorithm}{Algorithm}[section]
%\newtheorem{cor}{Corollary}
\newcommand{\BEQA}{\begin{eqnarray}}
\newcommand{\EEQA}{\end{eqnarray}}
\newcommand{\define}{\stackrel{\triangle}{=}}

\bibliographystyle{IEEEtran}
%\bibliographystyle{ieeetr}


\providecommand{\mbf}{\mathbf}
\providecommand{\pr}[1]{\ensuremath{\Pr\left(#1\right)}}
\providecommand{\qfunc}[1]{\ensuremath{Q\left(#1\right)}}
\providecommand{\sbrak}[1]{\ensuremath{{}\left[#1\right]}}
\providecommand{\lsbrak}[1]{\ensuremath{{}\left[#1\right.}}
\providecommand{\rsbrak}[1]{\ensuremath{{}\left.#1\right]}}
\providecommand{\brak}[1]{\ensuremath{\left(#1\right)}}
\providecommand{\lbrak}[1]{\ensuremath{\left(#1\right.}}
\providecommand{\rbrak}[1]{\ensuremath{\left.#1\right)}}
\providecommand{\cbrak}[1]{\ensuremath{\left\{#1\right\}}}
\providecommand{\lcbrak}[1]{\ensuremath{\left\{#1\right.}}
\providecommand{\rcbrak}[1]{\ensuremath{\left.#1\right\}}}
\theoremstyle{remark}
\newtheorem{rem}{Remark}
\newcommand{\sgn}{\mathop{\mathrm{sgn}}}
\providecommand{\abs}[1]{\left\vert#1\right\vert}
\providecommand{\res}[1]{\Res\displaylimits_{#1}} 
\providecommand{\norm}[1]{\left\lVert#1\right\rVert}
%\providecommand{\norm}[1]{\lVert#1\rVert}
\providecommand{\mtx}[1]{\mathbf{#1}}
\providecommand{\mean}[1]{E\left[ #1 \right]}
\providecommand{\fourier}{\overset{\mathcal{F}}{ \rightleftharpoons}}
%\providecommand{\hilbert}{\overset{\mathcal{H}}{ \rightleftharpoons}}
\providecommand{\system}{\overset{\mathcal{H}}{ \longleftrightarrow}}
	%\newcommand{\solution}[2]{\textbf{Solution:}{#1}}
\newcommand{\solution}{\noindent \textbf{Solution: }}
\newcommand{\cosec}{\,\text{cosec}\,}
\providecommand{\dec}[2]{\ensuremath{\overset{#1}{\underset{#2}{\gtrless}}}}
\newcommand{\myvec}[1]{\ensuremath{\begin{pmatrix}#1\end{pmatrix}}}
\newcommand{\mydet}[1]{\ensuremath{\begin{vmatrix}#1\end{vmatrix}}}
%\numberwithin{equation}{section}
\numberwithin{equation}{subsection}
%\numberwithin{problem}{section}
%\numberwithin{definition}{section}
\makeatletter
\@addtoreset{figure}{problem}
\makeatother

\let\StandardTheFigure\thefigure
\let\vec\mathbf
%\renewcommand{\thefigure}{\theproblem.\arabic{figure}}
\renewcommand{\thefigure}{\theproblem}
%\setlist[enumerate,1]{before=\renewcommand\theequation{\theenumi.\arabic{equation}}
%\counterwithin{equation}{enumi}


%\renewcommand{\theequation}{\arabic{subsection}.\arabic{equation}}

\def\putbox#1#2#3{\makebox[0in][l]{\makebox[#1][l]{}\raisebox{\baselineskip}[0in][0in]{\raisebox{#2}[0in][0in]{#3}}}}
     \def\rightbox#1{\makebox[0in][r]{#1}}
     \def\centbox#1{\makebox[0in]{#1}}
     \def\topbox#1{\raisebox{-\baselineskip}[0in][0in]{#1}}
     \def\midbox#1{\raisebox{-0.5\baselineskip}[0in][0in]{#1}}

\vspace{3cm}

\title{
	%\logo{
%Computational Approach to School Geometry
Vector Properties
%	}
}
\author{ G V V Sharma$^{*}$% <-this % stops a space
	\thanks{*The author is with the Department
		of Electrical Engineering, Indian Institute of Technology, Hyderabad
		502285 India e-mail:  gadepall@iith.ac.in. All content in this manual is released under GNU GPL.  Free and open source.}
	
}	
%\title{
%	\logo{Matrix Analysis through Octave}{\begin{center}\includegraphics[scale=.24]{tlc}\end{center}}{}{HAMDSP}
%}


% paper title
% can use linebreaks \\ within to get better formatting as desired
%\title{Matrix Analysis through Octave}
%
%
% author names and IEEE memberships
% note positions of commas and nonbreaking spaces ( ~ ) LaTeX will not break
% a structure at a ~ so this keeps an author's name from being broken across
% two lines.
% use \thanks{} to gain access to the first footnote area
% a separate \thanks must be used for each paragraph as LaTeX2e's \thanks
% was not built to handle multiple paragraphs
%

%\author{<-this % stops a space
%\thanks{}}
%}
% note the % following the last \IEEEmembership and also \thanks - 
% these prevent an unwanted space from occurring between the last author name
% and the end of the author line. i.e., if you had this:
% 
% \author{....lastname \thanks{...} \thanks{...} }
%                     ^------------^------------^----Do not want these spaces!
%
% a space would be appended to the last name and could cause every name on that
% line to be shifted left slightly. This is one of those "LaTeX things". For
% instance, "\textbf{A} \textbf{B}" will typeset as "A B" not "AB". To get
% "AB" then you have to do: "\textbf{A}\textbf{B}"
% \thanks is no different in this regard, so shield the last } of each \thanks
% that ends a line with a % and do not let a space in before the next \thanks.
% Spaces after \IEEEmembership other than the last one are OK (and needed) as
% you are supposed to have spaces between the names. For what it is worth,
% this is a minor point as most people would not even notice if the said evil
% space somehow managed to creep in.

%\WarningFilter{latex}{LaTeX Warning: You have requested, on input line 117, version}


% The paper headers
%\markboth{Journal of \LaTeX\ Class Files,~Vol.~6, No.~1, January~2007}%
%{Shell \MakeLowercase{\textit{et al.}}: Bare Demo of IEEEtran.cls for Journals}
% The only time the second header will appear is for the odd numbered pages
% after the title page when using the twoside option.
% 
% *** Note that you probably will NOT want to include the author's ***
% *** name in the headers of peer review papers.                   ***
% You can use \ifCLASSOPTIONpeerreview for conditional compilation here if
% you desire.




% If you want to put a publisher's ID mark on the page you can do it like
% this:
%\IEEEpubid{0000--0000/00\$00.00~\copyright~2007 IEEE}
% Remember, if you use this you must call \IEEEpubidadjcol in the second
% column for its text to clear the IEEEpubid mark.



% make the title area
\maketitle

\newpage

\tableofcontents

\bigskip

\renewcommand{\thefigure}{\theenumi}
\renewcommand{\thetable}{\theenumi}
%\renewcommand{\theequation}{\theenumi}

%\begin{abstract}
%%\boldmath
%In this letter, an algorithm for evaluating the exact analytical bit error rate  (BER)  for the piecewise linear (PL) combiner for  multiple relays is presented. Previous results were available only for upto three relays. The algorithm is unique in the sense that  the actual mathematical expressions, that are prohibitively large, need not be explicitly obtained. The diversity gain due to multiple relays is shown through plots of the analytical BER, well supported by simulations. 
%
%\end{abstract}
% IEEEtran.cls defaults to using nonbold math in the Abstract.
% This preserves the distinction between vectors and scalars. However,
% if the journal you are submitting to favors bold math in the abstract,
% then you can use LaTeX's standard command \boldmath at the very start
% of the abstract to achieve this. Many IEEE journals frown on math
% in the abstract anyway.

% Note that keywords are not normally used for peerreview papers.
%\begin{IEEEkeywords}
%Cooperative diversity, decode and forward, piecewise linear
%\end{IEEEkeywords}



% For peer review papers, you can put extra information on the cover
% page as needed:
% \ifCLASSOPTIONpeerreview
% \begin{center} \bfseries EDICS Category: 3-BBND \end{center}
% \fi
%
% For peerreview papers, this IEEEtran command inserts a page break and
% creates the second title. It will be ignored for other modes.
%\IEEEpeerreviewmaketitle

\begin{abstract}
This book provides a computational approach to school geometry based on the NCERT textbooks from Class 6-12.  Links to sample Python codes are available in the text.  
\end{abstract}

\section{Direction Vector}
\renewcommand{\theequation}{\theenumi}
%\begin{enumerate}[label=\arabic*.,ref=\theenumi]
\begin{enumerate}[label=\thesection.\arabic*.,ref=\thesection.\theenumi]
\numberwithin{equation}{enumi}
\item  Find the slope of a line, which passes through the origin, and the mid-point of the line segment joining the points $\vec{P} = \myvec{0\\ – 4}$ and $\vec{B} = \myvec{8\\ 0}$.
\solution 
%\input{./solutions/line_plane/24/solution.tex}
\item The slope of a line is double of the slope of another line. If the tangent of the angle
between them is $\frac{1}{3}$, find the slopes of the lines.
\solution 
%\input{./solutions/line_plane/25/solution.tex}
\item Find a unit vector that makes an angle of $90\degree, 60\degree$ and $30\degree$ with the positive x, y and z axis respectively.
%
\\
\solution
The direction vector is
%
\begin{align}
\label{eq:line_dir_cos}
\vec{x} &= \myvec{\cos 90\degree\\\cos 60\degree \\ \cos 30\degree} = \myvec{0 \\ \frac{1}{2}\\\frac{\sqrt{3}}{2}}
\end{align}
%
$\because \norm{\vec{x}}=1$, it is the desired unit vector.
\item Find the direction vectors and slopes of the lines passing through the points
%
\begin{enumerate}
\item \myvec{3\\-2} and \myvec{-1\\4}.
\item \myvec{3\\-2} and \myvec{7\\-2}.
\item \myvec{3\\-2} and \myvec{3\\4}.
\item Making an inclination of $60\degree$ with the positive direction of the x-axis.
\end{enumerate}
%
\solution
\begin{enumerate}
\item If the direction vector is 
\begin{align}
\myvec{1\\m}, 
\end{align}
%
the slope is $m$. Thus, the direction vector is
\begin{align}
\myvec{-1\\4} - \myvec{3\\-2} &= \myvec{-4\\6} = -\frac{1}{4} \myvec{-4\\6} 
\\
&=  \myvec{1\\-\frac{3}{2}} \implies m = -\frac{3}{2}
\end{align}
%
\item The direction vector is
\begin{align}
\myvec{7\\-2} - \myvec{3\\-2} &= \myvec{4\\0} 
\\
&=  \myvec{1\\0} \implies m = 0
\end{align}
%
\item The direction vector is
\begin{align}
\myvec{3\\4} - \myvec{3\\-2} &= \myvec{0\\6} 
\\
&=  \myvec{1\\ \infty} \implies m = \infty
\end{align}
%
\item The slope is $m = \tan 60 \degree = \sqrt{3}$ and the  direction vector is
\begin{align}
\myvec{1\\\sqrt{3}}
\end{align}
\end{enumerate}
\item If the angle between two lines is $\frac{\pi}{4}$ and the slope of one of the lines is $\frac{1}{4}$ find the slope of the other line.
\\
\solution The angle $\theta$ between two lines is given by 
%
\begin{align}
\tan \theta &= \frac{m_1-m_2}{1+m_1m_2}
\\
\implies 1 &= \frac{m_1-\frac{1}{4}}{1+\frac{m_1}{4}}
\\
\text{or } m_1 &= \frac{5}{3} 
\end{align}
%
%
%\\
%\solution 
%%Using \eqref{eq:tri_geo_ex_orth}
%\begin{align}
%\cbrak{\myvec{-2\\ 6}-\myvec{4\\8}}^T \cbrak{\myvec{8\\ 12}-\myvec{x\\24}}=  0 
%\end{align}
%%
%which can be used to obtain $x$.
\item If the points
$
\vec{A} = \myvec{6\\1}, 
\vec{B} = \myvec{8\\2}, 
\vec{C} = \myvec{9\\4}, 
\vec{D} = \myvec{p\\3}
$
are the vertices of a parallelogram, taken in order, find the value of $p$.
\\
\solution In the parallelogram $ABCD$, $AC$ and $BD$ bisect each other.  This can be used to find $p$.
\item Without using distance formula, show that points \myvec{– 2\\ – 1}, \myvec{4\\ 0}, \myvec{3\\ 3} and \myvec{–3\\ 2} are the vertices of a parallelogram.
\\
\solution
%\input{./solutions/5/chapters/quadrilateral/docq4.tex}

\item The two opposite vertices of a square are \myvec{-1\\2},  \myvec{3\\2}. Find the coordinates of the other two vertices.
\\
\solution
%\input{./solutions/7/chapters/quad/solution1.tex}

\item Find the direction vectors of the sides of a triangle with vertices
$
\vec{A} = \myvec{3\\5 \\-4},
\vec{B} = \myvec{-1\\1 \\2}, \text{ and }
\vec{C} = \myvec{-5\\ -5\\-2}
$
\\
\solution
%
%
\begin{align}
  \vec{A}-\vec{B} &= \myvec{0\\-1\\-2},
  \vec{A}-\vec{C} &= \myvec{-1\\-2\\0}
\end{align}
and 
\begin{align}
  \mydet{-1 & -2 \\-2 & 0} &= -4 \\
  \mydet{-2 & 0 \\0 & -1} &= 2 \\
  \mydet{0 & -1 \\-1 & -2} &= -1 \\
\end{align}
From   \eqref{eq:cross3d}, 
\begin{align}
  \frac{1}{2}\brak{\vec{A} - \vec{B}} \times \brak{\vec{A} - \vec{C}} = \frac{1}{2}\myvec{-4\\2\\-1}
\end{align}
and from \eqref{eq:cross3d}, the area of the triangle is
\begin{align}
  \frac{1}{2}\norm{\myvec{-4\\2\\-1}} = \frac{1}{2}\sqrt{4^2+2^2+1^2} = \frac{1}{2}\sqrt{21}
\end{align}

\item Find a unit vector in the direction of 
%
\begin{align}
\myvec{1\\1\\-2}.
\end{align}
%
\solution
%\input{solutions/aug/2/24.tex}

%
\item Find a unit vector in the direction of \myvec{2\\-1\\-2}.
\\
\solution
%\input{solutions/aug/2/21.tex}

%
%



\item Find a unit vector in the direction of the line passing through \myvec{-2\\4\\-5} and $\myvec{1\\2\\3}$.
%
\\
\solution
%\input{solutions/aug/2/22.tex}
\end{enumerate}
\section{Norm }
\renewcommand{\theequation}{\theenumi}
%\begin{enumerate}[label=\arabic*.,ref=\theenumi]
\begin{enumerate}[label=\thesection.\arabic*.,ref=\thesection.\theenumi]
\numberwithin{equation}{enumi}
\item Find the unit normal vector of the plane 
\begin{align}
\myvec{6 & -3 & -2}\vec{x}  = 1.
\end{align}
%
\solution The normal vector is 
%
\begin{align}
\vec{n} = \myvec{6 & -3 & -2}
\\
\because \norm{\vec{n}} = 7,
\end{align}
%
the unit normal vector is 
%
\begin{align}
\frac{\vec{n}}{\norm{\vec{n}}} = \frac{1}{7}\myvec{6 & -3 & -2}
\end{align}
%
%

\item Find the condition for $\vec{x} = \myvec{x_1\\x_2}$ to be equidistant from the points $\myvec{7\\1}, \myvec{3\\5}$.
\label{prob:line_perp_bisect}
%
\\
\solution From the given information,
%
\begin{align}
\norm{\vec{x}-\myvec{7\\1}}^2&=\norm{\vec{x}-\myvec{3\\5}}^2
\end{align}
\begin{multline}
\implies \norm{\vec{x}}^2 + \norm{\myvec{7\\1}}^2-2\myvec{7&1}\vec{x} 
\\= 
 \norm{\vec{x}}^2 + \norm{\myvec{3\\5}}^2-2\myvec{3&5}\vec{x} 
\end{multline}
%
which can be simplified to obtain
\begin{align}
\label{eq:line_p_bisect}
\myvec{1 & -1}\vec{x} = 2
\end{align}
%
which is the desired condition.  
The following code plots Fig. \ref{fig:line_perp_bisect}clearly showing that the above equation 
%\eqref{eq:line_p_bisect}
 is the perpendicular bisector of $AB$.

%
\begin{lstlisting}
codes/line/line_perp_bisect.py
\end{lstlisting}
%
%\begin{figure}[!ht]
%\includegraphics[width=\columnwidth]{./line/figs/line_perp_bisect.eps}
%\caption{}
%\label{fig:line_perp_bisect}
%\end{figure}
\item Find a point on the $y$-axis which is equidistant from the points $\vec{A} = \myvec{6\\5}, \vec{B} = \myvec{-4\\3}$.
\\
\solution
%\input{solutions/july/25/main1.tex}
\item Find the equation of set of points $\vec{P}$ such that
\begin{align}
PA^2+PB^2 =2k^2,
\end{align}
%
\begin{align}
\vec{A} = \myvec{3\\4 \\5},
\vec{B} = \myvec{-1\\3 \\-7},
\end{align}
%
respectively.
%
%
\solution
%\input{solutions/su2021/2/25.tex}   
\item Find the equation of the set of points $\vec{P}$ such that its distances from the points
$
\vec{A}=\myvec{3\\4\\-5}, 
\vec{B}=\myvec{-2\\1\\4}
$
are equal. 
\\
\solution
%\input{solutions/su2021/2/29/main.tex}
\end{enumerate}
\end{document}


