\let\negmedspace\undefined
\let\negthickspace\undefined
%\RequirePackage{amsmath}
\documentclass[journal,12pt,twocolumn]{IEEEtran}
%
% \usepackage{setspace}
% \usepackage{gensymb}
%\doublespacing
%\singlespacing
%\usepackage{silence}
%Disable all warnings issued by latex starting with "You have..."
%\usepackage{graphicx}
%\usepackage{amssymb}
%\usepackage{relsize}
\usepackage[cmex10]{amsmath}
%\usepackage{amsthm}
%\interdisplaylinepenalty=2500
%\savesymbol{iint}
%\usepackage{txfonts}
%\restoresymbol{TXF}{iint}
%\usepackage{wasysym}
\usepackage{amsthm}
%\usepackage{iithtlc}
% \usepackage{mathrsfs}
% \usepackage{txfonts}
% \usepackage{stfloats}
% \usepackage{steinmetz}
% \usepackage{bm}
% \usepackage{cite}
% \usepackage{cases}
% \usepackage{subfig}
%\usepackage{xtab}
\usepackage{longtable}
%\usepackage{multirow}
%\usepackage{algorithm}
%\usepackage{algpseudocode}
\usepackage{enumitem}
 \usepackage{mathtools}
% \usepackage{tikz}
% \usepackage{circuitikz}
% \usepackage{verbatim}
%\usepackage{tfrupee}
\usepackage[breaklinks=true]{hyperref}
%\usepackage{stmaryrd}
%\usepackage{tkz-euclide} % loads  TikZ and tkz-base
%\usetkzobj{all}
\usepackage{listings}
    \usepackage{color}                                            %%
    \usepackage{array}                                            %%
    \usepackage{longtable}                                        %%
    \usepackage{calc}                                             %%
    \usepackage{multirow}                                         %%
    \usepackage{hhline}                                           %%
    \usepackage{ifthen}                                           %%
  %optionally (for landscape tables embedded in another document): %%
    \usepackage{lscape}     
% \usepackage{multicol}
% \usepackage{chngcntr}
%\usepackage{enumerate}

%\usepackage{wasysym}
%\newcounter{MYtempeqncnt}
\DeclareMathOperator*{\Res}{Res}
%\renewcommand{\baselinestretch}{2}
\renewcommand\thesection{\arabic{section}}
\renewcommand\thesubsection{\thesection.\arabic{subsection}}
\renewcommand\thesubsubsection{\thesubsection.\arabic{subsubsection}}

\renewcommand\thesectiondis{\arabic{section}}
\renewcommand\thesubsectiondis{\thesectiondis.\arabic{subsection}}
\renewcommand\thesubsubsectiondis{\thesubsectiondis.\arabic{subsubsection}}

% correct bad hyphenation here
\hyphenation{op-tical net-works semi-conduc-tor}
\def\inputGnumericTable{}                                 %%

\lstset{
%language=C,
frame=single, 
breaklines=true,
columns=fullflexible
}
%\lstset{
%language=tex,
%frame=single, 
%breaklines=true
%}

\begin{document}
%


\newtheorem{theorem}{Theorem}[section]
\newtheorem{problem}{Problem}
\newtheorem{proposition}{Proposition}[section]
\newtheorem{lemma}{Lemma}[section]
\newtheorem{corollary}[theorem]{Corollary}
\newtheorem{example}{Example}[section]
\newtheorem{definition}[problem]{Definition}
%\newtheorem{thm}{Theorem}[section] 
%\newtheorem{defn}[thm]{Definition}
%\newtheorem{algorithm}{Algorithm}[section]
%\newtheorem{cor}{Corollary}
\newcommand{\BEQA}{\begin{eqnarray}}
\newcommand{\EEQA}{\end{eqnarray}}
\newcommand{\define}{\stackrel{\triangle}{=}}

\bibliographystyle{IEEEtran}
%\bibliographystyle{ieeetr}


\providecommand{\mbf}{\mathbf}
\providecommand{\pr}[1]{\ensuremath{\Pr\left(#1\right)}}
\providecommand{\qfunc}[1]{\ensuremath{Q\left(#1\right)}}
\providecommand{\sbrak}[1]{\ensuremath{{}\left[#1\right]}}
\providecommand{\lsbrak}[1]{\ensuremath{{}\left[#1\right.}}
\providecommand{\rsbrak}[1]{\ensuremath{{}\left.#1\right]}}
\providecommand{\brak}[1]{\ensuremath{\left(#1\right)}}
\providecommand{\lbrak}[1]{\ensuremath{\left(#1\right.}}
\providecommand{\rbrak}[1]{\ensuremath{\left.#1\right)}}
\providecommand{\cbrak}[1]{\ensuremath{\left\{#1\right\}}}
\providecommand{\lcbrak}[1]{\ensuremath{\left\{#1\right.}}
\providecommand{\rcbrak}[1]{\ensuremath{\left.#1\right\}}}
\theoremstyle{remark}
\newtheorem{rem}{Remark}
\newcommand{\sgn}{\mathop{\mathrm{sgn}}}
\providecommand{\abs}[1]{\left\vert#1\right\vert}
\providecommand{\res}[1]{\Res\displaylimits_{#1}} 
\providecommand{\norm}[1]{\left\lVert#1\right\rVert}
%\providecommand{\norm}[1]{\lVert#1\rVert}
\providecommand{\mtx}[1]{\mathbf{#1}}
\providecommand{\mean}[1]{E\left[ #1 \right]}
\providecommand{\fourier}{\overset{\mathcal{F}}{ \rightleftharpoons}}
%\providecommand{\hilbert}{\overset{\mathcal{H}}{ \rightleftharpoons}}
\providecommand{\system}{\overset{\mathcal{H}}{ \longleftrightarrow}}
	%\newcommand{\solution}[2]{\textbf{Solution:}{#1}}
\newcommand{\solution}{\noindent \textbf{Solution: }}
\newcommand{\cosec}{\,\text{cosec}\,}
\providecommand{\dec}[2]{\ensuremath{\overset{#1}{\underset{#2}{\gtrless}}}}
\newcommand{\myvec}[1]{\ensuremath{\begin{pmatrix}#1\end{pmatrix}}}
\newcommand{\mydet}[1]{\ensuremath{\begin{vmatrix}#1\end{vmatrix}}}
%\numberwithin{equation}{section}
\numberwithin{equation}{subsection}
%\numberwithin{problem}{section}
%\numberwithin{definition}{section}
\makeatletter
\@addtoreset{figure}{problem}
\makeatother

\let\StandardTheFigure\thefigure
\let\vec\mathbf
%\renewcommand{\thefigure}{\theproblem.\arabic{figure}}
\renewcommand{\thefigure}{\theproblem}
%\setlist[enumerate,1]{before=\renewcommand\theequation{\theenumi.\arabic{equation}}
%\counterwithin{equation}{enumi}


%\renewcommand{\theequation}{\arabic{subsection}.\arabic{equation}}

\def\putbox#1#2#3{\makebox[0in][l]{\makebox[#1][l]{}\raisebox{\baselineskip}[0in][0in]{\raisebox{#2}[0in][0in]{#3}}}}
     \def\rightbox#1{\makebox[0in][r]{#1}}
     \def\centbox#1{\makebox[0in]{#1}}
     \def\topbox#1{\raisebox{-\baselineskip}[0in][0in]{#1}}
     \def\midbox#1{\raisebox{-0.5\baselineskip}[0in][0in]{#1}}

\vspace{3cm}

\title{
	Cross Product
}
\author{ G V V Sharma$^{*}$% <-this % stops a space
	\thanks{*The author is with the Department
		of Electrical Engineering, Indian Institute of Technology, Hyderabad
		502285 India e-mail:  gadepall@iith.ac.in. All content in this manual is released under GNU GPL.  Free and open source.}
	
}	
%\title{
%	\logo{Matrix Analysis through Octave}{\begin{center}\includegraphics[scale=.24]{tlc}\end{center}}{}{HAMDSP}
%}


% paper title
% can use linebreaks \\ within to get better formatting as desired
%\title{Matrix Analysis through Octave}
%
%
% author names and IEEE memberships
% note positions of commas and nonbreaking spaces ( ~ ) LaTeX will not break
% a structure at a ~ so this keeps an author's name from being broken across
% two lines.
% use \thanks{} to gain access to the first footnote area
% a separate \thanks must be used for each paragraph as LaTeX2e's \thanks
% was not built to handle multiple paragraphs
%

%\author{<-this % stops a space
%\thanks{}}
%}
% note the % following the last \IEEEmembership and also \thanks - 
% these prevent an unwanted space from occurring between the last author name
% and the end of the author line. i.e., if you had this:
% 
% \author{....lastname \thanks{...} \thanks{...} }
%                     ^------------^------------^----Do not want these spaces!
%
% a space would be appended to the last name and could cause every name on that
% line to be shifted left slightly. This is one of those "LaTeX things". For
% instance, "\textbf{A} \textbf{B}" will typeset as "A B" not "AB". To get
% "AB" then you have to do: "\textbf{A}\textbf{B}"
% \thanks is no different in this regard, so shield the last } of each \thanks
% that ends a line with a % and do not let a space in before the next \thanks.
% Spaces after \IEEEmembership other than the last one are OK (and needed) as
% you are supposed to have spaces between the names. For what it is worth,
% this is a minor point as most people would not even notice if the said evil
% space somehow managed to creep in.

%\WarningFilter{latex}{LaTeX Warning: You have requested, on input line 117, version}


% The paper headers
%\markboth{Journal of \LaTeX\ Class Files,~Vol.~6, No.~1, January~2007}%
%{Shell \MakeLowercase{\textit{et al.}}: Bare Demo of IEEEtran.cls for Journals}
% The only time the second header will appear is for the odd numbered pages
% after the title page when using the twoside option.
% 
% *** Note that you probably will NOT want to include the author's ***
% *** name in the headers of peer review papers.                   ***
% You can use \ifCLASSOPTIONpeerreview for conditional compilation here if
% you desire.




% If you want to put a publisher's ID mark on the page you can do it like
% this:
%\IEEEpubid{0000--0000/00\$00.00~\copyright~2007 IEEE}
% Remember, if you use this you must call \IEEEpubidadjcol in the second
% column for its text to clear the IEEEpubid mark.



% make the title area
\maketitle

\newpage

\tableofcontents

\bigskip

\renewcommand{\thefigure}{\theenumi}
\renewcommand{\thetable}{\theenumi}
%\renewcommand{\theequation}{\theenumi}

%\begin{abstract}
%%\boldmath
%In this letter, an algorithm for evaluating the exact analytical bit error rate  (BER)  for the piecewise linear (PL) combiner for  multiple relays is presented. Previous results were available only for upto three relays. The algorithm is unique in the sense that  the actual mathematical expressions, that are prohibitively large, need not be explicitly obtained. The diversity gain due to multiple relays is shown through plots of the analytical BER, well supported by simulations. 
%
%\end{abstract}
% IEEEtran.cls defaults to using nonbold math in the Abstract.
% This preserves the distinction between vectors and scalars. However,
% if the journal you are submitting to favors bold math in the abstract,
% then you can use LaTeX's standard command \boldmath at the very start
% of the abstract to achieve this. Many IEEE journals frown on math
% in the abstract anyway.

% Note that keywords are not normally used for peerreview papers.
%\begin{IEEEkeywords}
%Cooperative diversity, decode and forward, piecewise linear
%\end{IEEEkeywords}



% For peer review papers, you can put extra information on the cover
% page as needed:
% \ifCLASSOPTIONpeerreview
% \begin{center} \bfseries EDICS Category: 3-BBND \end{center}
% \fi
%
% For peerreview papers, this IEEEtran command inserts a page break and
% creates the second title. It will be ignored for other modes.
%\IEEEpeerreviewmaketitle

\begin{abstract}
This manual provides an introduction to the cross product, based on the NCERT textbooks from Class 6-12.  
\end{abstract}

\section{Cross}
\renewcommand{\theequation}{\theenumi}
%\begin{enumerate}[label=\arabic*.,ref=\theenumi]
\begin{enumerate}[label=\thesection.\arabic*.,ref=\thesection.\theenumi]
\numberwithin{equation}{enumi}
%\renewcommand{\theequation}{\theenumi}
%\begin{enumerate}[label=\thesubsection.\arabic*.,ref=\thesubsection.\theenumi]
%\numberwithin{equation}{enumi}
%
%
\item Find the area of a parallelogram whose adjacent sides are determined by the vectors $\vec{a} = \myvec{1\\-1\\3}$ and $\vec{b}=\myvec{2\\-7\\1}$.
\\
\solution
%\input{solutions/aug/2/27.tex}
\item The vertices of $\triangle ABC$ are $\vec{A}=\myvec{4\\6},  \vec{B}=\myvec{1\\5}$ and  $\vec{C} =  \myvec{7\\2}$.  A line is drawn to intersect sides $AB$ and $AC$ at $D$ and $E$ respectively, such that
\begin{align}
\frac{AD}{AB}=\frac{AE}{AC}= \frac{1}{4}
\end{align}
%
Find 
\begin{align}
\frac{\text{area of }\triangle ADE}{\text{area of }\triangle ABC}.
\end{align}
\solution
%\input{solutions/aug/2/1/Assignment1.tex}
\item Find the scalar and vector products of the two vectors
\begin{align}
\vec{a} = \myvec{3\\-4\\5}, 
\vec{b} = \myvec{-2\\1\\-3}
\end{align}
%
\\
\solution 
%\input./solutions/point_vector/29/solution.tex}
\item Find the torque of a force \myvec{7\\3\\-5}
about the origin. The force
acts on a particle whose position vector is \myvec{1\\-1\\1}.
\\
\solution 
%\input./solutions/point_vector/30/Latex/solution.tex}
\item Given
\begin{align}
\vec{a}=\myvec{2\\1\\3},
\vec{b}=\myvec{3\\5\\-2},
\end{align}
find $\norm{\vec{a} \times \vec{b}}$.
%
\\
\solution Use \eqref{eq:tri_cross_prod}.
\item Find area of the triangle with vertices at the point given in each of the following :\\
(i) \myvec{1&0}, \myvec{6&0}, \myvec{4&3}\\
(ii) \myvec{2&7}, \myvec{1&1}, \myvec{10&8}\\
(iii) \myvec{-2&-3}, \myvec{3&2}, \myvec{-1&-8}\\
\solution 
\begin{enumerate}
    \item 
    %\input{solutions/det/73/1/main.tex}
    

\end{enumerate}
\item Find values of k if area of triangle is 4sq.units and vertices are \\
(i)) \myvec{k&0}, \myvec{4&0}, \myvec{0&2} \\ (ii) \myvec{-2&0}, \myvec{0&4}, \myvec{0&k}
\item If the area of triangle is 35 sq.units with vertices \myvec{2&-6}, \myvec{5&4} and \myvec{k&4}.then k is 
\begin{enumerate}
\item 12
\item -2
\item -12,-2
\item 12,-2
\end{enumerate}
\solution
%\input{solutions/aug/1/77.tex}
\item Find the area of a triangle having the points 
$
\vec{A} = \myvec{1\\1 \\1},
\vec{B} = \myvec{1\\2 \\3}, \text{ and }
\vec{C} = \myvec{2\\ 3\\1}
$
as its vertices.
\\
\solution
%\input{solutions/su2021/2/7.tex}
%
\item Find the area of a triangle with vertices
$
\vec{A} = \myvec{1\\1 \\2},
\vec{B} = \myvec{2\\3 \\5}, \text{ and }
\vec{C} = \myvec{1\\ 5\\5}
$
\\
\solution
%\input{solutions/su2021/2/8.tex}
%
\item Find the area of the triangle whose vertices are
\begin{enumerate}
\item \myvec{2\\3}, \myvec{-1\\0},  \myvec{2\\-4}
\item  \myvec{-5\\-1},  \myvec{3\\-5},  \myvec{5\\2}
\end{enumerate}
\solution
%\input{./solutions/6/chapters/triangle/solution.tex}

\item Find the area of the triangle formed by joining the mid points of the sides of a triangle whose vertices are  \myvec{0\\-1},  \myvec{2\\1},  \myvec{0\\3}.
\\
\solution
%\input{./solutions/7/chapters/triangle/solution.tex}
\item Verify that the median of $\triangle ABC$ with vertices $\vec{A}=\myvec{4\\-6},  \vec{B}=\myvec{3\\-2}$ and  $\vec{C} =  \myvec{5\\2}$ divides it into two triangles of equal areas.
\\
\solution
%\input{./solutions/8/chapters/solution.tex}
\item Find the area of a triangle whose vertices are 
$\vec{A}=\myvec{1\\-1}, 
\vec{B} = \myvec{-4\\6}$ and
$ 
\vec{C} = \myvec{-3\\-5}
$.
%
\\
\solution
  Using Hero's formula, the following code computes the area of the  triangle as 24.
%
\begin{lstlisting}
codes/triangle/area_tri.py
\end{lstlisting}
%
%
\item Find the area of a triangle formed by the vertices $\vec{A}=\myvec{5\\2}, \vec{B}=\myvec{4\\7}, \vec{C}=\myvec{7\\-4}$.
%\\
\solution  The area of $\triangle ABC$ is also obtained  in terms of the  {\em magnitude} of the determinant of the matrix $\vec{M}$ in  \eqref{eq:tri_geo_ex_diff_mat} as
%
\begin{align}
\frac{1}{2}\mydet{\vec{M}}
\end{align}
The computation is done in \textbf{area\_tri.py}
\item Find the area of a triangle formed by the points $\vec{P}=\myvec{-1.5\\3}, \vec{Q}=\myvec{6\\-2}, \vec{R}=\myvec{-3\\4}$.
\\
\solution Another formula for the area of $\triangle ABC$  is
%
\begin{align}
\frac{1}{2}\mydet{1 & 1 & 1\\ \vec{A} & \vec{B} & \vec{C} }
\end{align}
%
\item Find the area of a triangle having the points
%
\begin{align}
\vec{A} = \myvec{1\\1 \\1},
\vec{B} = \myvec{1\\2 \\3},
\vec{C} = \myvec{2\\ 3\\1}
\end{align}
%
as its vertices.
\\
\solution The area of a triangle using the {\em vector product} is obtained as
\begin{align}
\frac{1}{2}\norm{\brak{\vec{B}-\vec{A}}\times \brak{\vec{C}-\vec{A}}}
\end{align}
%
For any two vectors $\vec{a}=\myvec{a_1\\a_2\\a_3}, \vec{b}=\myvec{b_1\\b_2\\b_3}$, 
\begin{align}
\label{eq:tri_cross_prod}
\vec{a}\times \vec{b} = \myvec{0 & -a_3 & a_2 \\ a_3 & 0 & -a_1 \\ -a_2 & a_1 & 0}\myvec{b_1\\b_2\\b_3}
\end{align}
%
The following code computes the area using the vector product.
%
\begin{lstlisting}
codes/triangle/area_tri_vec.py
\end{lstlisting}
%
\item Find the area of a parallelogram whose adjacent sides are given by the vectors \myvec{3\\1\\4} and \myvec{1\\-1\\1}.
%
\\
\solution  The area is given by 
%
\begin{align}
\frac{1}{2}\norm{\myvec{3\\1\\4} \times \myvec{1\\-1\\1}}
\end{align}
\item Draw a quadrilateral in the Cartesian plane, whose vertices are \myvec{– 4\\ 5}, \myvec{0\\ 7}, \myvec{5\\ – 5} and \myvec{– 4\\ –2}. Also, find its area.
\\
\solution
%\input{./solutions/3/chapters/quad/solution.tex}

\item Find the area of a rhombus if its vertices are 
\begin{align}
\vec{P} &= \myvec{3\\0}, \vec{Q} =\myvec{4\\5},
\\
\vec{R} &= \myvec{-1\\4}, \vec{S} = \myvec{-2\\-1} 
\end{align}
taken in order.
\\
\solution
%\input{./solutions/4/chapters/quadrilateral/solution.tex}

\item  Find the area of the quadrilateral whose vertices, taken in order, are 
 \myvec{-4\\2},  \myvec{-3\\-5},  \myvec{3\\-2},  \myvec{2\\3}. 
\\
\solution
%\input{./solutions/6/chapters/quadrilateral/solution.tex}
\item Find the area of a parallelogram whose adjacent sides are given by the vectors \myvec{3\\1\\4} and \myvec{1\\-1\\1}.
\\
\solution
%\input{./solutions/quad/8/chapters/solution.tex}
\item Find the area of a rectangle $ABCD$ with vertices
$\vec{A} = \myvec{-1\\\frac{1}{2}\\ 4},
 \vec{B} = \myvec{1\\\frac{1}{2}\\ 4},
\vec{C} = \myvec{1\\-\frac{1}{2}\\ 4},
\vec{D} = \myvec{-1\\-\frac{1}{2}\\ 4}.
$
\\
\solution
%\input{./solutions/quad/10/solution.tex}
\item The two adjacent sides of a parallelogram are \myvec{2\\ -4 \\ -5} and  \myvec{1\\-2\\ -3}. Find the unit vector parallel to its diagonal.  Also, find its area.
%
\\
\solution
%\input{solutions/july/1/Assignment_3_Edited.tex}

\item If $\vec{A} = \myvec{-5\\7}, \vec{B} = \myvec{-4\\-5}, \vec{C} = \myvec{-1\\-6}, \vec{D} = \myvec{4\\5}$, find the area of the quadrilateral $ABCD$.
%
\\
\solution The area of  $ABCD$ is the sum of the areas of trianges ABD and CBD and is given by 
\begin{multline}
\frac{1}{2}\norm{\brak{\vec{A}-\vec{B}}\times \brak{\vec{A}-\vec{D}}}
\\
+
\frac{1}{2}\norm{\brak{\vec{C}-\vec{B}}\times \brak{\vec{C}-\vec{D}}}
\end{multline}
\end{enumerate}
\section{Application}
\renewcommand{\theequation}{\theenumi}
%\begin{enumerate}[label=\arabic*.,ref=\theenumi]
\begin{enumerate}[label=\thesection.\arabic*.,ref=\thesection.\theenumi]
\numberwithin{equation}{enumi}

\item If $\vec{a}^T\vec{b} = 0$ and $\vec{a}\times \vec{b}$ = 0, what can you conclude about $\vec{a}$ and $\vec{b}$?
\item Show that 
\begin{align}
\brak{\vec{a}-\vec{b}}\times \brak{\vec{a}+\vec{b}} = 2\brak{\vec{a}\times\vec{b}}
\end{align}
\item If $\norm{\vec{a}} = 3, \norm{\vec{b}} =\frac{\sqrt{2}}{3}$, then $\vec{a}\times \vec{b}$ is a unit vector if the angle between $\vec{a}$ and $\vec{b}$ is 
\begin{enumerate}[itemsep = 2pt]
\begin{multicols}{2}
\item $\frac{\pi}{6}$
\item $\frac{\pi}{4}$
\item $\frac{\pi}{3}$
\item $\frac{\pi}{2}$
\end{multicols}
\end{enumerate}
\end{enumerate}
\end{document}


