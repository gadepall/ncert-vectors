\let\negmedspace\undefined
\let\negthickspace\undefined
%\RequirePackage{amsmath}
\documentclass[journal,12pt,twocolumn]{IEEEtran}
%
% \usepackage{setspace}
 \usepackage{gensymb}
%\doublespacing
%\singlespacing
%\usepackage{silence}
%Disable all warnings issued by latex starting with "You have..."
%\usepackage{graphicx}
\usepackage{amssymb}
%\usepackage{relsize}
\usepackage[cmex10]{amsmath}
%\usepackage{amsthm}
%\interdisplaylinepenalty=2500
%\savesymbol{iint}
%\usepackage{txfonts}
%\restoresymbol{TXF}{iint}
%\usepackage{wasysym}
\usepackage{amsthm}
%\usepackage{iithtlc}
% \usepackage{mathrsfs}
% \usepackage{txfonts}
% \usepackage{stfloats}
% \usepackage{steinmetz}
 \usepackage{bm}
% \usepackage{cite}
% \usepackage{cases}
% \usepackage{subfig}
%\usepackage{xtab}
\usepackage{longtable}
%\usepackage{multirow}
%\usepackage{algorithm}
%\usepackage{algpseudocode}
\usepackage{enumitem}
% \usepackage{mathtools}
% \usepackage{tikz}
% \usepackage{circuitikz}
% \usepackage{verbatim}
%\usepackage{tfrupee}
\usepackage[breaklinks=true]{hyperref}
%\usepackage{stmaryrd}
%\usepackage{tkz-euclide} % loads  TikZ and tkz-base
%\usetkzobj{all}
\usepackage{listings}
    \usepackage{color}                                            %%
    \usepackage{array}                                            %%
    \usepackage{longtable}                                        %%
    \usepackage{calc}                                             %%
    \usepackage{multirow}                                         %%
    \usepackage{hhline}                                           %%
    \usepackage{ifthen}                                           %%
  %optionally (for landscape tables embedded in another document): %%
    \usepackage{lscape}     
% \usepackage{multicol}
% \usepackage{chngcntr}
%\usepackage{enumerate}

%\usepackage{wasysym}
%\newcounter{MYtempeqncnt}
\DeclareMathOperator*{\Res}{Res}
%\renewcommand{\baselinestretch}{2}
\renewcommand\thesection{\arabic{section}}
\renewcommand\thesubsection{\thesection.\arabic{subsection}}
\renewcommand\thesubsubsection{\thesubsection.\arabic{subsubsection}}

\renewcommand\thesectiondis{\arabic{section}}
\renewcommand\thesubsectiondis{\thesectiondis.\arabic{subsection}}
\renewcommand\thesubsubsectiondis{\thesubsectiondis.\arabic{subsubsection}}

% correct bad hyphenation here
\hyphenation{op-tical net-works semi-conduc-tor}
\def\inputGnumericTable{}                                 %%

\lstset{
%language=C,
frame=single, 
breaklines=true,
columns=fullflexible
}
%\lstset{
%language=tex,
%frame=single, 
%breaklines=true
%}

\begin{document}
%


\newtheorem{theorem}{Theorem}[section]
\newtheorem{problem}{Problem}
\newtheorem{proposition}{Proposition}[section]
\newtheorem{lemma}{Lemma}[section]
\newtheorem{corollary}[theorem]{Corollary}
\newtheorem{example}{Example}[section]
\newtheorem{definition}[problem]{Definition}
%\newtheorem{thm}{Theorem}[section] 
%\newtheorem{defn}[thm]{Definition}
%\newtheorem{algorithm}{Algorithm}[section]
%\newtheorem{cor}{Corollary}
\newcommand{\BEQA}{\begin{eqnarray}}
\newcommand{\EEQA}{\end{eqnarray}}
\newcommand{\define}{\stackrel{\triangle}{=}}

\bibliographystyle{IEEEtran}
%\bibliographystyle{ieeetr}


\providecommand{\mbf}{\mathbf}
\providecommand{\pr}[1]{\ensuremath{\Pr\left(#1\right)}}
\providecommand{\qfunc}[1]{\ensuremath{Q\left(#1\right)}}
\providecommand{\sbrak}[1]{\ensuremath{{}\left[#1\right]}}
\providecommand{\lsbrak}[1]{\ensuremath{{}\left[#1\right.}}
\providecommand{\rsbrak}[1]{\ensuremath{{}\left.#1\right]}}
\providecommand{\brak}[1]{\ensuremath{\left(#1\right)}}
\providecommand{\lbrak}[1]{\ensuremath{\left(#1\right.}}
\providecommand{\rbrak}[1]{\ensuremath{\left.#1\right)}}
\providecommand{\cbrak}[1]{\ensuremath{\left\{#1\right\}}}
\providecommand{\lcbrak}[1]{\ensuremath{\left\{#1\right.}}
\providecommand{\rcbrak}[1]{\ensuremath{\left.#1\right\}}}
\theoremstyle{remark}
\newtheorem{rem}{Remark}
\newcommand{\sgn}{\mathop{\mathrm{sgn}}}
\providecommand{\abs}[1]{\left\vert#1\right\vert}
\providecommand{\res}[1]{\Res\displaylimits_{#1}} 
\providecommand{\norm}[1]{\left\lVert#1\right\rVert}
%\providecommand{\norm}[1]{\lVert#1\rVert}
\providecommand{\mtx}[1]{\mathbf{#1}}
\providecommand{\mean}[1]{E\left[ #1 \right]}
\providecommand{\fourier}{\overset{\mathcal{F}}{ \rightleftharpoons}}
%\providecommand{\hilbert}{\overset{\mathcal{H}}{ \rightleftharpoons}}
\providecommand{\system}{\overset{\mathcal{H}}{ \longleftrightarrow}}
	%\newcommand{\solution}[2]{\textbf{Solution:}{#1}}
\newcommand{\solution}{\noindent \textbf{Solution: }}
\newcommand{\cosec}{\,\text{cosec}\,}
\providecommand{\dec}[2]{\ensuremath{\overset{#1}{\underset{#2}{\gtrless}}}}
\newcommand{\myvec}[1]{\ensuremath{\begin{pmatrix}#1\end{pmatrix}}}
\newcommand{\mydet}[1]{\ensuremath{\begin{vmatrix}#1\end{vmatrix}}}
%\numberwithin{equation}{section}
\numberwithin{equation}{subsection}
%\numberwithin{problem}{section}
%\numberwithin{definition}{section}
\makeatletter
\@addtoreset{figure}{problem}
\makeatother

\let\StandardTheFigure\thefigure
\let\vec\mathbf
%\renewcommand{\thefigure}{\theproblem.\arabic{figure}}
\renewcommand{\thefigure}{\theproblem}
%\setlist[enumerate,1]{before=\renewcommand\theequation{\theenumi.\arabic{equation}}
%\counterwithin{equation}{enumi}


%\renewcommand{\theequation}{\arabic{subsection}.\arabic{equation}}

\def\putbox#1#2#3{\makebox[0in][l]{\makebox[#1][l]{}\raisebox{\baselineskip}[0in][0in]{\raisebox{#2}[0in][0in]{#3}}}}
     \def\rightbox#1{\makebox[0in][r]{#1}}
     \def\centbox#1{\makebox[0in]{#1}}
     \def\topbox#1{\raisebox{-\baselineskip}[0in][0in]{#1}}
     \def\midbox#1{\raisebox{-0.5\baselineskip}[0in][0in]{#1}}

\vspace{3cm}

\title{
	%\logo{
%Computational Approach to School Geometry
Points and Vectors
%	}
}
\author{ G V V Sharma$^{*}$% <-this % stops a space
	\thanks{*The author is with the Department
		of Electrical Engineering, Indian Institute of Technology, Hyderabad
		502285 India e-mail:  gadepall@iith.ac.in. All content in this manual is released under GNU GPL.  Free and open source.}
	
}	
%\title{
%	\logo{Matrix Analysis through Octave}{\begin{center}\includegraphics[scale=.24]{tlc}\end{center}}{}{HAMDSP}
%}


% paper title
% can use linebreaks \\ within to get better formatting as desired
%\title{Matrix Analysis through Octave}
%
%
% author names and IEEE memberships
% note positions of commas and nonbreaking spaces ( ~ ) LaTeX will not break
% a structure at a ~ so this keeps an author's name from being broken across
% two lines.
% use \thanks{} to gain access to the first footnote area
% a separate \thanks must be used for each paragraph as LaTeX2e's \thanks
% was not built to handle multiple paragraphs
%

%\author{<-this % stops a space
%\thanks{}}
%}
% note the % following the last \IEEEmembership and also \thanks - 
% these prevent an unwanted space from occurring between the last author name
% and the end of the author line. i.e., if you had this:
% 
% \author{....lastname \thanks{...} \thanks{...} }
%                     ^------------^------------^----Do not want these spaces!
%
% a space would be appended to the last name and could cause every name on that
% line to be shifted left slightly. This is one of those "LaTeX things". For
% instance, "\textbf{A} \textbf{B}" will typeset as "A B" not "AB". To get
% "AB" then you have to do: "\textbf{A}\textbf{B}"
% \thanks is no different in this regard, so shield the last } of each \thanks
% that ends a line with a % and do not let a space in before the next \thanks.
% Spaces after \IEEEmembership other than the last one are OK (and needed) as
% you are supposed to have spaces between the names. For what it is worth,
% this is a minor point as most people would not even notice if the said evil
% space somehow managed to creep in.

%\WarningFilter{latex}{LaTeX Warning: You have requested, on input line 117, version}


% The paper headers
%\markboth{Journal of \LaTeX\ Class Files,~Vol.~6, No.~1, January~2007}%
%{Shell \MakeLowercase{\textit{et al.}}: Bare Demo of IEEEtran.cls for Journals}
% The only time the second header will appear is for the odd numbered pages
% after the title page when using the twoside option.
% 
% *** Note that you probably will NOT want to include the author's ***
% *** name in the headers of peer review papers.                   ***
% You can use \ifCLASSOPTIONpeerreview for conditional compilation here if
% you desire.




% If you want to put a publisher's ID mark on the page you can do it like
% this:
%\IEEEpubid{0000--0000/00\$00.00~\copyright~2007 IEEE}
% Remember, if you use this you must call \IEEEpubidadjcol in the second
% column for its text to clear the IEEEpubid mark.



% make the title area
\maketitle

\newpage

\tableofcontents

\bigskip

\renewcommand{\thefigure}{\theenumi}
\renewcommand{\thetable}{\theenumi}
%\renewcommand{\theequation}{\theenumi}

%\begin{abstract}
%%\boldmath
%In this letter, an algorithm for evaluating the exact analytical bit error rate  (BER)  for the piecewise linear (PL) combiner for  multiple relays is presented. Previous results were available only for upto three relays. The algorithm is unique in the sense that  the actual mathematical expressions, that are prohibitively large, need not be explicitly obtained. The diversity gain due to multiple relays is shown through plots of the analytical BER, well supported by simulations. 
%
%\end{abstract}
% IEEEtran.cls defaults to using nonbold math in the Abstract.
% This preserves the distinction between vectors and scalars. However,
% if the journal you are submitting to favors bold math in the abstract,
% then you can use LaTeX's standard command \boldmath at the very start
% of the abstract to achieve this. Many IEEE journals frown on math
% in the abstract anyway.

% Note that keywords are not normally used for peerreview papers.
%\begin{IEEEkeywords}
%Cooperative diversity, decode and forward, piecewise linear
%\end{IEEEkeywords}



% For peer review papers, you can put extra information on the cover
% page as needed:
% \ifCLASSOPTIONpeerreview
% \begin{center} \bfseries EDICS Category: 3-BBND \end{center}
% \fi
%
% For peerreview papers, this IEEEtran command inserts a page break and
% creates the second title. It will be ignored for other modes.
%\IEEEpeerreviewmaketitle

\begin{abstract}
This manual provides an introduction to vectors and their properties,  based on the CBSE question papers, year 2020,  from Class 10 and 12.  
\end{abstract}

\section{Definitions}
\subsection{$2\times 1$ vectors}
\renewcommand{\theequation}{\theenumi}
%\begin{enumerate}[label=\arabic*.,ref=\theenumi]
\begin{enumerate}[label=\thesubsection.\arabic*.,ref=\thesubsection.\theenumi]
\numberwithin{equation}{enumi}
\item Let 
\begin{align}
  \vec{A} \equiv \overrightarrow{A} &= \myvec{a_1\\a_2} 
  \\
  &\equiv a_1\overrightarrow{i}+a_2\overrightarrow{j}, 
  \\
  \vec{B} &= \myvec{b_1\\b_2}, 
\end{align}
be $2 \times 1$ vectors.
Then, the determinant of the $2 \times 2$ matrix 
\begin{align}  
  \vec{M} = \myvec{\vec{A} & \vec{B}}
\end{align}
is defined as
\begin{align}
  \label{eq:det2d}
  \mydet{\vec{M}} &= \mydet{\vec{A} & \vec{B}} 
  \\
  &= \mydet{a_1 & b_1\\a_2 & b_2} = a_1b_2 - a_2 b_1
\end{align}
%
\item The value of the cross product of two vectors is given by  
  \eqref{eq:det2d}.
\item The area of the triangle with vertices $\vec{A}, \vec{B}, \vec{C}$ is given by the absolute value of 
\begin{align}
  \label{eq:area2d}
\frac{1}{2} \mydet{\vec{A-B} & \vec{A-C}}
  \end{align}
  \item  The transpose of $\vec{A}$ is defined as
\begin{align}
  \label{eq:transpose2d}
  \vec{A}^{\top}  = \myvec{a_1 & a_2}
\end{align}
%
\item The {\em inner product} or {\em dot product} is defined as
\begin{align}
  \label{eq:dot2d}
  \vec{A}^{\top} \vec{B} &\equiv \vec{A} \cdot \vec{B} 
  \\
  &= \myvec{a_1 & a_2} \myvec{b_1 \\ b_2}= a_1b_1+a_2b_2 
\end{align}
%
\item {\em norm} of $\vec{A}$ is defined as
\begin{align}
  \label{eq:norm2d}
  \norm{A} &\equiv \mydet{\overrightarrow{A}}
  \\
  &= \sqrt{\vec{A}^{\top} \vec{A}}= \sqrt{a_1^2+a_2^2}
\end{align}
Thus, 
\begin{align}
%  \label{eq:norm2d}
  \norm{\lambda \vec{A}} &\equiv \mydet{\lambda\overrightarrow{A}}
  \\
  &= \abs{\lambda} \norm{\vec{A}}
\end{align}
\item The distance betwen the points $\vec{A}$ and $\vec{B}$ is given by 
\begin{align}
  \label{eq:norm2d_dist}
\norm{\vec{A}-\vec{B}} 
\end{align}
\item Let $\vec{x}$ be equidistant from the points $\vec{A}$ and $\vec{B}$.  Then 
\begin{align}
	\norm{\vec{x}-\vec{A}} &=
\norm{\vec{A}-\vec{B}} 
\\
	\implies \norm{\vec{x}-\vec{A}}^2 &=
\norm{\vec{x}-\vec{B}}^2 
\end{align}
which can be expressed as 
\begin{multline}
%  \label{eq:norm2d_dist}
	\brak{\vec{x}-\vec{A}}^{\top} \brak{\vec{x}-\vec{A}}=
	\brak{\vec{x}-\vec{B}}^{\top} 
\brak{\vec{x}-\vec{B}}
\\
	\implies	\norm{\vec{x}}^2-2{\vec{x}}^{\top}\vec{A} + \norm{\vec{A}}^2
	\\= \norm{\vec{x}}^2-2{\vec{x}}^{\top}\vec{B} + \norm{\vec{B}}^2
\end{multline}
which can be simplified to obtain
  \begin{align}
	  \brak{\vec{A}-\vec{B}}^{\top}{\vec{x}} 
	  =  \frac{\norm{\vec{A}}^2 - \norm{\vec{B}}^2}{2}
  \label{eq:norm2d_equidist}
  \end{align}
  \item The angle between two vectors is given by 
  \begin{align}
    \label{eq:angle2d}
    \theta = \cos^{-1}\frac{\vec{A}^{\top} \vec{B}}{\norm{A}\norm{B}}
  \end{align}
  \item If two vectors are orthogonal (perpendicular), 
  \begin{align}
    \label{eq:angle2d_orth}
\vec{A}^{\top} \vec{B} = 0
  \end{align}

  \item The {\em direction vector} of the line joining two points $\vec{A},\vec{B}$ is given by 
  \begin{align}
    \label{eq:dir_vec}
    \vec{m} = \vec{A}-\vec{B}
  \end{align}
\item The unit vector in the direction of $\vec{m}$ is defined as
\begin{align}
    \frac{\vec{m}}{\norm{\vec{m}}}
\end{align}
\item If the direction vector of a line is expressed as 
	\begin{align}
    \vec{m} = \myvec{1\\m},
\end{align}
 the $m$ is defined to be the {\em} slope of the line. 
  \item The {\em normal vector} to $\vec{m}$ is defined by 
  \begin{align}
    \label{eq:normal_vec}
    \vec{m}^{\top}  \vec{n} = 0
  \end{align}
  \item The point $\vec{P}$ that divides the line segment $AB$ in the ratio $k:1$  is given by 

  \begin{align}
	  \vec{P}&= \frac{k\vec{B}+ \vec{A}}{k+1}
	  \label{eq:section_formula}
  \end{align}
\item  The standard basis vectors are defined as 

  \begin{align}
  \vec{e}_1&= \myvec{1\\0}, 
  \vec{e}_2&= \myvec{0\\1} 
  \end{align}
\end{enumerate}
\subsection{$3\times 1$ vectors}
\renewcommand{\theequation}{\theenumi}
%\begin{enumerate}[label=\arabic*.,ref=\theenumi]
\begin{enumerate}[label=\thesubsection.\arabic*.,ref=\thesubsection.\theenumi]
\numberwithin{equation}{enumi}

\item Let 
\begin{align}
  \vec{A} &= \myvec{a_1\\a_2 \\ a_3} \equiv a_1\overrightarrow{i}+a_2\overrightarrow{j}+a_3\overrightarrow{j}, 
  \\
  \vec{B} &= \myvec{b_1\\b_2 \\ b_3}, 
\end{align}
and 
\begin{align}
  \vec{A}_{ij} &= \myvec{a_i\\a_j}, 
  \vec{B}_{ij} &= \myvec{b_i\\b_j}, 
\end{align}

\item The {\em cross product} or {\em vector product} of $\vec{A}, \vec{B}$ is defined as
\begin{align}
  \label{eq:cross3d}
	\vec{A} \times \vec{B} = \myvec{ \mydet{\vec{A}_{23} & \vec{B}_{23}} \\ \mydet{\vec{A}_{31} & \vec{B}_{31}} \\ \mydet{\vec{A}_{12}  & \vec{B}_{12}}}
\end{align}
\item Verify that
\begin{align}
  \vec{A} \times \vec{B} = -  \vec{B} \times \vec{A} 
\end{align}

\end{enumerate}
\subsection{Eigenvalues and Eigenvectors}
\renewcommand{\theequation}{\theenumi}
%\begin{enumerate}[label=\arabic*.,ref=\theenumi]
\begin{enumerate}[label=\thesubsection.\arabic*.,ref=\thesubsection.\theenumi]
\numberwithin{equation}{enumi}
\item The eigenvalue $\lambda$ and the eigenvector $\vec{x}$  for a matrix $\vec{A}$ are defined as, 
\begin{align}
  \vec{A} \vec{x} = \lambda \vec{x}
\end{align}
\item The eigenvalues are calculated by solving the
equation
\begin{align}
  \label{eq:chareq}
f\brak{\lambda} = \mydet{\lambda \vec{I}- \vec{A} } =0
\end{align}
The above equation is known as the characteristic equation.
\item According to the Cayley-Hamilton theorem,
\begin{align}
	\label{eq:cayley}
  f(\lambda) = 0 \implies f\brak{\vec{A}} = 0
\end{align}
\item The trace of a square  matrix is defined to be the sum of the diagonal elements.
\begin{align}
	\label{eq:trace}
	\text{tr}\brak{\vec{A}}=\sum_{i=1}^{N}a_{ii}.
\end{align}
	where $a_{ii}$ is the $i$th diagonal element of the matrix $\vec{A}$. 	
\item The trace of a matrix is equal to the sum of the eigenvalues
\begin{align}
	\label{eq:trace_eig}
	\text{tr}\brak{\vec{A}}=\sum_{i=1}^{N}\lambda_i
\end{align}


\end{enumerate}
\subsection{Determinants}
\renewcommand{\theequation}{\theenumi}
%\begin{enumerate}[label=\arabic*.,ref=\theenumi]
\begin{enumerate}[label=\thesubsection.\arabic*.,ref=\thesubsection.\theenumi]
\numberwithin{equation}{enumi}

\item Let 
\begin{align}
	\vec{A} = \myvec{a_1 & b_1 & c_1  \\ a_2 & b_2 & c_2  \\ a_3 & b_3 & c_3}.
\end{align}
be a $3 \times 3$ matrix. 
Then, 
\begin{multline}
	\mydet{\vec{A}} = a_1 \myvec{ b_2 & c_2 \\  b_3 & c_3} - a_2\myvec{ b_1 & c_1 \\  b_3 & c_3 }  \\ + a_3\myvec{a_1 & b_1 \\ a_2 & b_2 }.
\end{multline}
\item Let $\lambda_1,\lambda_2, \dots, \lambda_n$ be the eigenvalues of a matrix $\vec{A}$.  Then,   the product of the eigenvalues is equal to the determinant of $\vec{A}$.
\begin{align}
	\mydet{\vec{A}} = \prod_{i=1}^{n}\lambda_i
\end{align}
%
\item 
\begin{align}
	\mydet{\vec{A}\vec{B}} = \mydet{\vec{A}}\mydet{\vec{B}}
\end{align}
\item If $\vec{A}$ be an $n \times n$ matrix, 
\begin{align}
	\mydet{k\vec{A}} = k^n\mydet{\vec{A}}
\end{align}

\end{enumerate}
\subsection{Rank of a Matrix}
\renewcommand{\theequation}{\theenumi}
%\begin{enumerate}[label=\arabic*.,ref=\theenumi]
\begin{enumerate}[label=\thesubsection.\arabic*.,ref=\thesubsection.\theenumi]
\numberwithin{equation}{enumi}
\item The rank of a matrix is defined as the number of linearly independent rows.  This is also known as the row rank.
\item Row rank = Column rank.
\item The rank of a matrix is obtained as the number of nonzero rows obtained after row reduction.
\item An $n \times n$ matrix is invertible if and only if its rank is $n$.
\end{enumerate}
\subsection{Inverse of a Matrix}
\renewcommand{\theequation}{\theenumi}
%\begin{enumerate}[label=\arabic*.,ref=\theenumi]
\begin{enumerate}[label=\thesubsection.\arabic*.,ref=\thesubsection.\theenumi]
\numberwithin{equation}{enumi}
\item For a $2 \times 2$ matrix 
\begin{align}
	\vec{A} = \myvec{a_1 & b_1  \\ a_2 & b_2 },
\end{align}
the inverse is given by 
\begin{align}
	\vec{A}^{-1} = \frac{1}{\mydet{\vec{A}}}\myvec{b_2 & -b_1  \\ -a_2 & a_1 },
\end{align}
\item For higher order matrices, the inverse should be calculated using row operations.
\end{enumerate}
\section{Geometry}
\subsection{Two Dimensions}
\renewcommand{\theequation}{\theenumi}
%\begin{enumerate}[label=\arabic*.,ref=\theenumi]
\begin{enumerate}[label=\thesubsection.\arabic*.,ref=\thesubsection.\theenumi]
\numberwithin{equation}{enumi}
\item The equation of a line  is given by  
\begin{align}
	\label{eq:normal_line}
   \vec{n}^{\top}\vec{x} = c
\end{align}
		where $\vec{n}$ is the normal vector of the line.
	\item The equation of a line with normal vector $\vec{n}$ and passing through a point $\vec{A}$ 
		is given by 
\begin{align}
	\label{eq:normal_line_pt}
	\vec{n}^{\top}\brak{\vec{x}-\vec{A}} =0 
\end{align}
\item The parametric equation of a line  is given by  
\begin{align}
	\label{eq:dir_line}
	\vec{x} = \vec{A} + \lambda \vec{m}
\end{align}
		where $\vec{m}$ is the direction vector of the line and $\vec{A}$ is any point on the line.
	\item The distance from a point $\vec{P}$ to the line  in 
	\eqref{eq:normal_line}
	is given by 
\begin{align}
	\label{eq:line_dist_2d}
	d = \frac{\abs{   \vec{n}^{\top}\vec{P}-c }}{\norm{\vec{n}}}	
\end{align}
		\solution Without loss of generality, let $\vec{A}$ be the foot of the perpendicular from $\vec{P}$ to the line in 
	\eqref{eq:dir_line}.  The equation of the normal to 
	\eqref{eq:normal_line} can then be expressed as 
\begin{align}
	\label{eq:dir_line_normal_dist}
	\vec{x} &= \vec{A} + \lambda \vec{n}
	\\
	\implies 
	\vec{P}- \vec{A} &=  \lambda \vec{n}
	\label{eq:dir_line_normal_dist_pa}
\end{align}
$\because \vec{P}$ lies on 
		\eqref{eq:dir_line_normal_dist}.
From the above, the desired distance can be expressed as 
\begin{align}
d = 	\norm{\vec{P}- \vec{A}}= \abs{\lambda} \norm{\vec{n}}
	\label{eq:dir_line_normal_dist_pa_d}
\end{align}
From 
	\eqref{eq:dir_line_normal_dist_pa},
\begin{align}
	\vec{n}^{\top}
	\brak{\vec{P}- \vec{A}} &=  \lambda \vec{n}^{\top}\vec{n} = \lambda\norm{\vec{n}}^2
	\\
	\implies \abs{\lambda}&= \frac{\abs{\vec{n}^{\top}
	\brak{\vec{P}- \vec{A}}}}{\norm{\vec{n}}^2} 
\end{align}
	Substituting the above in \eqref{eq:dir_line_normal_dist_pa_d} and using 
	the fact that 
\begin{align}
   \vec{n}^{\top}\vec{A} = c
\end{align}
from 	\eqref{eq:normal_line}, yields 
	\eqref{eq:line_dist_2d}.

	\item The distance from the origin to the line  in 
	\eqref{eq:normal_line}
	is given by 
\begin{align}
	\label{eq:dist_line_2d_orig}
	d = \frac{\abs{   c }}{\norm{\vec{n}}}	
\end{align}
\item The distance between the parallel lines 
\begin{align}
	\label{eq:parallel_lines}
	\begin{split}
		\vec{n}^{\top}\vec{x} &= c_1
		\\
		\vec{n}^{\top}\vec{x} &= c_2
	\end{split}
\end{align}
is given by 
\begin{align}
	\label{eq:dist_lines_2d}
	d = \frac{\abs{   c_1-c_2 }}{\norm{\vec{n}}}	
\end{align}
\item The equation of the line perpendicular to 
	\eqref{eq:normal_line}
		and passing through the point $\vec{P}$ is given by 
\begin{align}
	\vec{m}^{\top}\brak{\vec{x}-\vec{P}}  = 0
\end{align}
\item The foot of the perpendicular from $\vec{P}$ to the line in 
	\eqref{eq:normal_line}
	is given by 
\begin{align}
	\label{eq:normal_line_foot}
	\myvec{ \vec{m} & \vec{n}}^{\top}\vec{x}= \myvec{\vec{m}^{\top}\vec{P}\\ c }  
\end{align}
% 
\solution From
	\eqref{eq:normal_line} and 
	\eqref{eq:normal_line_pt} 
the foot of the perpendicular satisfies the equations 
\begin{align}
	\vec{n}^{\top}\vec{x} &= c
	\\
	\vec{m}^{\top}\brak{\vec{x}-\vec{P} }&=0 
\end{align}
where $\vec{m}$ is the direction vector of the given line.  Combining the above into a matrix equation results in 
	\eqref{eq:normal_line_foot}.
\end{enumerate}
 
\subsection{Three Dimensions}
\renewcommand{\theequation}{\theenumi}
%\begin{enumerate}[label=\arabic*.,ref=\theenumi]
\begin{enumerate}[label=\thesubsection.\arabic*.,ref=\thesubsection.\theenumi]
\numberwithin{equation}{enumi}
\item The area of a triangle with vertices $\vec{A}, \vec{B}, \vec{C}$ is given by 
\begin{align}
  \label{eq:area3d}
 \frac{1}{2} \norm{\brak{\vec{A} - \vec{B}} \times \brak{\vec{A} - \vec{C}}}
\end{align}

\item Points $\vec{A}, \vec{B}, \vec{C}$ are on a line if 
\begin{align}
  \label{eq:line_rank}
  \text{rank}\myvec{\vec{A} \\ \vec{B} \\ \vec{C} }  = 1
\end{align}
\item Points $\vec{A}, \vec{B}, \vec{C}, \vec{D}$ form a paralelogram if 
\begin{align}
  \label{eq:parallelgm_rank}
  \text{rank}\myvec{\vec{A} \\ \vec{B} \\ \vec{C} \\ \vec{D}  }  = 1, 
  \text{rank}\myvec{\vec{A} \\ \vec{B} \\ \vec{C} }  = 2
\end{align}
\item The equation of a line  is given by  
	\eqref{eq:dir_line}
	\item The equation of a plane is given by
	\eqref{eq:normal_line}
	\item The distance from the origin to the line  in 
	\eqref{eq:normal_line}
	is given by 
	\eqref{eq:dist_line_2d_orig}
\item The equation of the line perpendicular to 
	\eqref{eq:normal_line}
		and passing through the point $\vec{P}$ is given by 
\begin{align}
	\vec{m}^{\top}\brak{\vec{x}-\vec{P}}  = 0
\end{align}
\item The foot of the perpendicular from $\vec{P}$ to the line in 
	\eqref{eq:normal_line}
	is given by 
\begin{align}
	\myvec{ \vec{m} & \vec{n}}^{\top}\vec{x}= \myvec{\vec{m}^{\top}\vec{P}\\ c }  
\end{align}
% 
\item The distance from a point $\vec{P}$  to the line in 
	\eqref{eq:dir_line} is given by 
\begin{align}
	\label{dist_3d_def_final}
		d = \norm{\vec{A} -\vec{P}}^2 - \frac{\cbrak{\vec{m}^{\top}\brak{\vec{A}-\vec{P} 
	}}^2}{\norm{\vec{m}}^2}
%	d =\norm{\vec{A}  -\vec{P}
% -\frac{\vec{m}^{\top}\brak{\vec{A} 
%			-\vec{P}}}
%			{ \norm{\vec{m}}^2}
%	\vec{m}}
		\end{align}
		\solution 
\begin{align}
	\label{dist_3d_def}
	d\brak{\lambda } &=\norm{\vec{A} + \lambda \vec{m}-\vec{P}}
	\\
\implies 	d^2\brak{\lambda } &=\norm{\vec{A} + \lambda \vec{m}-\vec{P}}^2
\end{align}
which can be simplified to obtain 
	\begin{multline}
d^2\brak{\lambda } =\lambda^2 \norm{\vec{m}}^2+2\lambda \vec{m}^{\top}\brak{\vec{A} 
		-\vec{P}}
		\\
		+\norm{\vec{A} -\vec{P}}^2
	\end{multline}
which is of the form 
\begin{align}
	\label{dist_3d_def_quad}
	d^2\brak{\lambda } &=a \lambda^2 + 2b\lambda +c
	\\
	&=a \cbrak{\brak{\lambda+ \frac{b}{a}}^2 +\sbrak{\frac{c}{a}-\brak{\frac{b}{a}}^2 }}
\end{align}
with 
\begin{align}
	\label{dist_3d_def_quad_abc}
	a = \norm{\vec{m}}^2, b = \vec{m}^{\top}\brak{\vec{A} 
		-\vec{P}}, c = 
		\norm{\vec{A} -\vec{P}}^2
\end{align}
which can be expressed as 
%		\begin{multline}
%			d^2\brak{\lambda } =\norm{\vec{m}}^2\brak{\lambda + \frac{\vec{m}^{\top}\brak{\vec{A}-\vec{P} }}{\vec{m}}^2}}^2 +2\lambda \vec{m}^{\top}\brak{\vec{A} 
%			-\vec{P}}
%			\\
%			+\norm{\vec{A} -\vec{P}}^2
%		\end{multline}
		From the above, $d^2\brak{\lambda}$ is smallest when upon substituting from 
	\eqref{dist_3d_def_quad_abc}
\begin{align}
	\label{dist_3d_def_quad_small}
	\lambda+ \frac{b}{2a} &= 0 \implies \lambda = - \frac{b}{2a}
	&= -\frac{\vec{m}^{\top}\brak{\vec{A} 
			-\vec{P}}}
			{ \norm{\vec{m}}^2}
	%		\label{dist_3d_lam}
\end{align}
and consequently, 
\begin{align}
	d_{\min}\brak{\lambda } &=a \brak{\frac{c}{a}-\brak{\frac{b}{a}}^2 } 
	\\
	&=c - \frac{b^2}{a }
\end{align}
yielding
	\eqref{dist_3d_def_final} after substituting from 
	\eqref{dist_3d_def_quad_abc}.
%From 	\eqref{dist_3d_def} and \eqref{dist_3d_lam}, 
%	\eqref{dist_3d_def} is obtained.
\item The distance between the parallel planes 
	\eqref{eq:parallel_lines}
	is given by 
	\eqref{eq:dist_lines_2d}.
\item The plane 
		\begin{align}
		\vec{n}^{\top}
			-\vec{x} = c
			\label{eq:plain_contain}
		\end{align}
		contains the line 
		\begin{align}
			\vec{x} = \vec{A}+\lambda \vec{m}
			\label{eq:line_contain}
		\end{align}
		if 
		\begin{align}
		\vec{m}^{\top}\vec{n} = 0
			\label{eq:line_plain_contain}
		\end{align}
		\solution Any point on the line 
			\eqref{eq:line_contain}
			should also satisfy 
			\eqref{eq:plain_contain}.  Hence, 
		\begin{align}
			\vec{n}^{\top}\brak{\vec{A}+\lambda \vec{m}} &= \vec{n}^{\top}\vec{A}=c
		\end{align}
		which can be simplified to obtain
			\eqref{eq:line_plain_contain}
		\item Let a plane pass through the points $\vec{A},\vec{B}$ and be perpendicular to the plane 
		\begin{align}
		\vec{n}^{\top}\vec{x} =c 
			\label{eq:plane_3d_2pt_perp_given}
		\end{align}
		Then the equation of this plane is given by 
		\begin{align}
		\vec{p}^{\top}\vec{x} = 1
			\label{eq:plane_3d_2pt}
		\end{align}
		where
		\begin{align}
			\vec{p} = 		\myvec{\vec{A} & \vec{B} & \vec{n}}^{-\top}  \myvec{1 \\ 1 \\ 0}
			\label{eq:plane_3d_2pt_perp_norm}
		\end{align}
	\solution From the given information, 
		\begin{align}
			\vec{p}^{\top}\vec{A} &=d 
			\\
			\vec{p}^{\top}\vec{B} &=d 
			\\
			\vec{p}^{\top}\vec{n} &= 0
			\label{eq:plane_3d_2pt_perp_system}
		\end{align}
		$\because$ the normal vectors to the two planes will also be perpendicular.  The system of equations in 
			\eqref{eq:plane_3d_2pt_perp_system}
			can be expressed as the matrix equation
		\begin{align}
			\myvec{\vec{A} & \vec{B} & \vec{n}}^{\top}\vec{p} = d\myvec{1 \\ 1 \\ 0}
			\label{eq:plane_3d_2pt_perp_system_temp}
		\end{align}
		which yields 
			\eqref{eq:plane_3d_2pt_perp_norm}
			upon normalising with $d$.
		\item The intersection of the line represented by 
	\eqref{eq:dir_line}
	with the plane represented by 
	\eqref{eq:normal_line}
	is given by 
\begin{align}
	\label{eq:dir_line_plane_isect}
	\vec{x} &= \vec{A} + \frac{c - \vec{n}^{\top}\vec{A}}{\vec{n}^{\top}\vec{m}}
\vec{m}
\end{align}
\solution From 
	\eqref{eq:dir_line}
	and 
	\eqref{eq:normal_line},
\begin{align}
	\vec{x} &= \vec{A} + \lambda \vec{m}
	\\
	\vec{n}^{\top}\vec{x} &= c
	\\
	\implies 
	\vec{n}^{\top}\brak{\vec{A} + \lambda \vec{m}}&= c
	\label{eq:dir_line_plane_inter}
\end{align}
which can be simplified to obtain
\begin{align}
	\vec{n}^{\top}\vec{A} + \lambda 	\vec{n}^{\top}\vec{m}&= c
	\\
	\implies \lambda &= \frac{c - \vec{n}^{\top}\vec{A}}{\vec{n}^{\top}\vec{m}}
\end{align}
Substituting the above in 
	\eqref{eq:dir_line_plane_inter}
	yields
	\eqref{eq:dir_line_plane_isect}.
\item The foot of the perpendicular from the point $\vec{P}$ to the line  represented by 
	\eqref{eq:dir_line}
	is given by 
\begin{align}
	\label{eq:plane_line_foot_ans}
	\vec{x} &= \vec{A} + \frac{ \vec{m}^{\top}\brak{\vec{P} - \vec{A}}}{\norm{\vec{m}}^2}
\vec{m}
\end{align}
\solution  Let the equation of the line be 
\begin{align}
	\label{eq:dir_line_foot}
	\vec{x} &= \vec{A} + \lambda \vec{m}
\end{align}
	The equation of the plane perpendicular to the given line passing through $\vec{P}$ is given by
\begin{align}
	\label{eq:plane_line_foot}
	\vec{m}^{\top}\brak{\vec{x}-\vec{P}}  &= 0
	\\
	\implies \vec{m}^{\top}\vec{x}  &= \vec{m}^{\top}\vec{P}
\end{align}
The desired foot of the perpendicular is the intersection of 
	\eqref{eq:dir_line_foot} with 
	\eqref{eq:plane_line_foot}
	which can be obtained from 
	\eqref{eq:dir_line_plane_isect}
	as 
	\eqref{eq:plane_line_foot_ans}
\item The foot of the perpendicular from a point $\vec{P}$ to a plane is $\vec{Q}$.  The equation of the plane is given by 
\begin{align}
	\label{eq:plane_foot_perp}
	\brak{\vec{P}-\vec{Q}}^{\top}\brak{\vec{x}-\vec{Q}} = 0
\end{align}
	\solution  The normal vector to the plane is given by 
\begin{align}
	\vec{n}= \vec{P}-\vec{Q} 
\end{align}
	Hence, the equation of the plane is
	\eqref{eq:plane_foot_perp}.
	
%\renewcommand{\theequation}{\theenumi}
%%\begin{enumerate}[label=\arabic*.,ref=\theenumi]
%\begin{enumerate}[label=\thesubsection.\arabic*.,ref=\thesubsection.\theenumi]
%\numberwithin{equation}{enumi}
%
\item (Affine Transformation) Let $\vec{A},\vec{C}$, be opposite sides of a square. The other two points can be obtained as  
\begin{align}
  \label{eq:square_points}
  \vec{B} = \frac{\norm{\vec{A}-\vec{C}}}{\sqrt{2}} \vec{P}\vec{e}_1+\vec{A}
  \\
  \vec{D} = \frac{\norm{\vec{A}-\vec{C}}}{\sqrt{2}} \vec{P}\vec{e}_2+\vec{A}
\end{align}
where 
\begin{align}
	\vec{P} = \myvec{\cos \brak{\theta-\frac{\pi}{4}} & \sin  \brak{\theta-\frac{\pi}{4}} \\ \sin \brak{\theta-\frac{\pi}{4}} & \cos \brak{\theta-\frac{\pi}{4}}}
\end{align}
and 
\begin{align}
	\cos\theta = \frac{\brak{\vec{C}-\vec{A}}^{\top}\vec{e}_1}{\norm{\vec{A}-\vec{C}}\norm{\vec{e}_1}}
\end{align}
%\end{enumerate}
\end{enumerate}
\section{Class 10}
\renewcommand{\theequation}{\theenumi}
%\begin{enumerate}[label=\arabic*.,ref=\theenumi]
\begin{enumerate}[label=\thesection.\arabic*.,ref=\thesection.\theenumi]
\numberwithin{equation}{enumi}
\item  Find the distance between the points $\myvec{m \\ -n}$ and $\myvec{-m \\ n}$
	\\
		\solution Letting 
		\begin{align}
			\vec{A} &= \myvec{m \\ -n}, \vec{B}=\myvec{-m \\ n}
			\\
			\vec{A}-\vec{B} &= 2\myvec{m \\ -n}
			\\
			\implies \norm{\vec{A}-\vec{B}} &=2\norm{\myvec{m \\ -n}}
			\\
			&=2 \sqrt{\myvec{m & -n}\myvec{m \\ -n}} 
\\
			&			= 2 \sqrt{m^2+n^2}
		\end{align}
	\item Find a point on the $x-$axis which is equidistant from $\myvec{ -4 \\ 0}$ and $\myvec{10 \\ 0}$
\\
\solution Letting the given points be $\vec{A},\vec{B}$. 
		\begin{align}
	\because  \vec{A}-\vec{B} &= \myvec{ -4 \\ 0}-\myvec{10 \\ 0}
	  \\
	  &= \myvec{ -14 \\ 0},
	  \\
			\text{and }	   \frac{\norm{\vec{A}}^2 - \norm{\vec{B}}^2}{2} &= -42,
  \end{align}
  \eqref{eq:norm2d_equidist}, can be expressed as 
  \begin{align}
	  \myvec{ -14 & 0}\vec{x} 
	  &=  42
	  \\
	  \implies 
	  \myvec{ -14 & 0}\myvec{x \\ 0} 
	  &=  -42
	  \\
	  \text{or, } x =3 
  \end{align}
  Hence, the desired point is $\myvec{ 3 \\ 0}$.
  \item Find the centre of a circle whose end points of a diameter are $\myvec{ -6 \\ 3}$ and $\myvec{6 \\ 4}$.
	  \\
		\solution 
Using section formula, 
	  from \eqref{eq:section_formula},
		the desired point is given by 
  \begin{align}
	  \vec{O}&= \frac{\vec{B}+ \vec{A}}{2}
	  \\
	  &= \frac{1}{2}\sbrak{\myvec{ -6 \\ 3}+\myvec{6 \\ 4}}
	  \\
	  &=\frac{1}{2}\myvec{0 \\ 7}
  \end{align}
  \item $AOBC$  is a rectangle whose three vertices are $A = \myvec{0\\-3}, O = \myvec{0\\0}, B = \myvec{4\\0}$.  Find the length of its diagonal. 
	  \\
		\solution The fourth point is given by 
  \begin{align}
	 OC =  \vec{C}&=\myvec{4 \\ -3}
  \end{align}
  The length of the diagonal is 
		\begin{align}
			\norm{\vec{C}} &= 
			 \sqrt{\myvec{4 & -3}\myvec{4 \\ -3}} 
\\
			&			=  \sqrt{25} = 5
		\end{align}
	\item  Find the ratio in which the $y$-axis divides the line segment joining the points $\myvec{6 \\ -4}, \myvec{-2 \\ -7} $.  Also find the point of intersection.
		\\
		\solution  Let the desired point on the $y$-axis be  
		\begin{align}
\vec{P} = \myvec{0 & y}
		\end{align}
Using section formula, 
	  from \eqref{eq:section_formula},
		\begin{align}
			\vec{P} &= \myvec{0 & y}= \frac{1}{k+1}\sbrak{\myvec{6 \\ -4}+ k\myvec{-2 \\ -7} }
			\\
			&	\implies 6 - 2k = 0 \text{ or, }k = 3
		\end{align}
		Also, 
		\begin{align}
			y &= \frac{-4-7k}{k+1}
			\\
			&=-\frac{25}{4}
		\end{align}
		Thus, the desired point is  $ -\frac{25}{4}
\myvec{0 \\ 1 }$.
	\item   Show that the points $\myvec{7 \\ 10}, \myvec{-2 \\ 5} $ and $\myvec{3 \\ -4}$ are vertices of an isoscles right triangle.
		\\
		\solution Let the given points be $\vec{A}, \vec{B}, \vec{C}$ respectively. 
Then, the direction vectors of $AB, BC$ and $CA$ are
		\begin{align}
			\vec{A} -\vec{B}&= \myvec{7 \\ 10} -\myvec{-2 \\ 5} = \myvec{9 \\ 5}
			\\
			\vec{B} -\vec{C}&=  -\myvec{-2 \\ 5}-\myvec{3 \\ -4} = \myvec{-5 \\ 9}
			\\
			\vec{C} -\vec{A}&= \myvec{3 \\ -4} -\myvec{7 \\ 10} = \myvec{-4 \\ -14}
		\end{align}
		From the above,  we find that 
		\begin{align}
			\brak{\vec{A} -\vec{B}}^{\top}\brak{\vec{B} -\vec{C}}&=  \myvec{9 & 5}\myvec{-5 \\ 9}
			\\
			&=0
			\\
			\brak{\vec{B} -\vec{C}}^{\top}\brak{\vec{C} -\vec{A}}&=  \myvec{-5 & 9}\myvec{-4 \\ -14}
\\
			&=-106
			\\
			\brak{\vec{C} -\vec{A}}^{\top}\brak{\vec{A} -\vec{B}}&=  \myvec{-4 & -14}\myvec{9 \\ 5}
\\
			&=-106
		\end{align}
		From  the above equations, 
    \eqref{eq:angle2d} and 
    \eqref{eq:angle2d_orth},
		\begin{align}
			\brak{\vec{A} -\vec{B}}\perp \brak{\vec{B} -\vec{C}}
			\\
			\angle BCA = 
			\angle CAB  
		\end{align}
		Thus, the triangle is right angled and isosceles.
\end{enumerate}
\section{Class 12}
\renewcommand{\theequation}{\theenumi}
%\begin{enumerate}[label=\arabic*.,ref=\theenumi]
\begin{enumerate}[label=\thesection.\arabic*.,ref=\thesection.\theenumi]
\numberwithin{equation}{enumi}
	\item Find the area of a triangle formed by vertices O, A and B, where 
		\begin{align}
			\vec{A} = \myvec{1 \\ 2 \\ 3},
			\vec{B}  = \myvec{-3 \\ -2 \\ 1},
		\end{align}
\solution $\because $
		\begin{align}
			\myvec{2 & -2 \\ 3 & 1 } &= 8,
\\
			\myvec{3 & 1 \\ 1 & -3 } &= -10,
			\\
			\myvec{1 & -3 \\ 2 & -2 } &= 4,
		\end{align}
		\begin{align}
			\vec{A}  \times 
			\vec{B}  = \myvec{8 \\ -10 \\ 4},
		\end{align}
		and the  desired area can be obtained from 
  \eqref{eq:cross3d} and 
  \eqref{eq:area3d} as
		\begin{align}
			\frac{1}{2}\norm{\myvec{8 \\ -10 \\ 4}} = 3 \sqrt{5}
		\end{align}
\item  The coordinates of the foot of the perpendicular drawn from the point $ \left(2,-3,4 \right) $ on the y-axis is 
\begin{enumerate}
    \item $\left(2,3,4\right)$
    \item $\left(-2,-3,-4\right)$
    \item $\left(0,-3,0\right)$
    \item $\left(2,0,4\right)$
\end{enumerate}
\item  The angle between the vectors $ \hat{i} - \hat{j} $ and $ \hat{j} - \hat{k} $ is
\begin{enumerate}
    \item $\frac{-\pi}{3}$
    \item 0
    \item $\frac{\pi}{3}$
    \item $\frac{2\pi}{3}$
\end{enumerate}
\item  If A is a non-singular square matrix of order 3 such that $ A^2 =3A $, then value of  $\begin{vmatrix}A \end{vmatrix}$ is
\begin{enumerate}
    \item -3
     \item 3
     \item 9
     \item 27
\end{enumerate}
\item  If $\begin{vmatrix}\overrightarrow{a} \end{vmatrix} = 4 $ and  $ -3\leq \lambda \leq 2 $ then $\begin{vmatrix}\lambda \overrightarrow{a} \end{vmatrix} $ lies in
\begin{enumerate}
    \item $\left[0,12\right]$
    \item $\left[2,3\right]$
    \item $\left[8,12\right]$
    \item $\left[-12,8\right]$
\end{enumerate}
\item  If $\begin{vmatrix} 2x & -9 \\ -2 & x \end{vmatrix}$ = $\begin{vmatrix} -4 & 8 \\ 1 & -2 \end{vmatrix}$ , then value of x is -------------
\item  The corner points of the feasible region of an LPP are (0,0),(0,8),(2,7),(5,4) and (6,0). The maximum profit P=3x+2y occurs at the point .................
\item  The distance between parallel planes 2x+y-2z-6=0 and 4x+2y-4z=0 is ------------------ units.
\begin{center}
        OR
    \end{center}
    If P(1,0,-3) is the foot of the perpendicular from the origin to the plane, then the Cartesian equation of the plane is .................... 
    
\item  Find the coordinates of the point where the line $\frac{x-1}{3}$ = $\frac{y+4}{7}$ = $\frac{z+4}{2}$ cuts the xy-plane.
\item  Find a vector $\overrightarrow{r}$ equally inclined to the three axes and whose magnitude is $3\sqrt{3}$ units.
 
    
\item    Find the angle between unit vectors $\overrightarrow{a}$ and $\overrightarrow{b}$ so that $\sqrt{3}$ $\overrightarrow{a}$ - $\overrightarrow{b}$ is also a unit vector.
\item If $A=\begin{bmatrix}-3 & 2 \\ 1 & -1 \end{bmatrix} $ and $ I=\begin{bmatrix}1 & 0 \\ 0 & 1 \end{bmatrix}$, Find scalar k so that $A^2 + I = kA$.
\item Show that the plane $x-5y-2z=1$ contains the line $\frac{x-5}{3}=y=2-z$. 
\item A cottage industry manufactures pedestal lamps and wooden shades. Both the products require machine time as well as craftsman time in the making. The number of hours required for producing 1 unit of each and the corresponding profit is given in the following table 
%\end{center}
In a day, the factory has availability of not more than 42 hours of machine time and 24 hours of craftsman time.
Assuming that all items manufactured are sold, how should the manufacturer schedule his daily production in order to maximise the profit? Formulate it as an LPP and solve it graphically.
	\begin{table}[!ht]
		\centering
%\begin{center}
\resizebox{\columnwidth}{!}{
\begin{tabular}{|c|c|c|c|}
\hline
% \begin{tabularx}{\linewidth} {lX}
 Item & Machine Time & Craftsman Time & Profit(in INR) \\ 
 \hline
 Pedestal Lamp & 1.5 hours & 3 hours & 30\\  
 \hline
 Wooden shades & 3 hours & 1 hour & 20 \\
 \hline
%  \end{tabularx}
\end{tabular}
}
	\caption{}
	\label{table}
\end{table}
\item Find the equation of the plane passing through the points $(1,0,-2)$ , $(3,-1,0)$ and perpendicular to the plane $ 2x-y+z=8 $. Also find the distance of the plane thus obtained from the origin.  
    \item Using integration, Find the area lying above x-axis and included between the circle $ x^2 + y^2 =8x $ and inside the parabola $ y^2 =4x $.   
    
    \item   Using the method of integration, find the area of the triangle ABC, coordinates of whose vertices are A(2,0), B(4,5) and C(6,3). 
\item If $A=\begin{bmatrix}5 & -1 & 4 \\ 2 & 3 & 5 \\ 5 & -2 & 6 \end{bmatrix} $, Find $A^{-1}$ and use it to solve the following system of the equations: \\
\begin{center}
        $5x-y+4z=5 $\\
        $2x+3y+5z=2 $\\
        $5x-2y+6z=-1 $
    \end{center}
    
\item If x,y,z are different and $\begin{vmatrix}x & x^2 & 1+x^3 \\ y & y^2 & 1+y^3 \\ z & z^2 & 1+z^3 \end{vmatrix} =0 $, then using properties of determinants show that $1+xyz=0.$  
\end{enumerate}
\section{JEE}
\renewcommand{\theequation}{\theenumi}
%\begin{enumerate}[label=\arabic*.,ref=\theenumi]
\begin{enumerate}[label=\thesection.\arabic*.,ref=\thesection.\theenumi]
\numberwithin{equation}{enumi}
\item Find the locus of $\vec{z}$ if 
\begin{align}
	\vec{e}_1^{\top}\brak{\frac{\vec{z}-\vec{e}_1}{2\vec{z}+\vec{e}_2}} = 1
\end{align}
\item Let $\alpha$ be a root of the equation
\begin{align}
	x^2+x+1 = 0
\end{align}
and the matrix 
\begin{align}
	\vec{A} = 	\frac{1}{\sqrt{3}}\myvec{1 & 1 & 1\\ 1 & \alpha &  \alpha^2 \\ 1 & \alpha^2  & \alpha}^4 
\end{align}
		Find $\vec{A}^31$
	\item The line 
\begin{align}
	\myvec{m & -1}\vec{x} = -4
\end{align}
is a tangent to the parabolas
\begin{align}
	\vec{x}^{\top}\myvec{0 & 0 \\ 0 &  1}\vec{x} - 4 \myvec{1 & 0}\vec{x}&= 0
	\\
	\vec{x}^{\top}\myvec{1 & 0 \\ 0  & 0}\vec{x} - 2 \myvec{0 & b}\vec{x}&= 0
\end{align}
Find the value of $b$.
\item If the distance between the foci of an ellipse is 6 and the distance between its directrices is
	12, then find the length of its latus rectum.
\item Find the area of the region, enclosed by the circle 
\begin{align}
	\vec{x}^{\top}\vec{x} = 2
\end{align}
which is not common to the region bounded by the parabola 
\begin{align}
	\vec{x}^{\top}\myvec{0 & 0 \\ 0 &  1}\vec{x} -  \myvec{1 & 0}\vec{x}= 0
\end{align}
and the straight line
\begin{align}
\myvec{1 & -1 }\vec{x}  = 0.
\end{align}
\item Let $P$ be a plane passing through the points 
\begin{align}
	\myvec{2 \\  1 \\ 0 }, 	\myvec{4 \\  1 \\ 1 }, 	\myvec{5 \\  0 \\ 1 }
\end{align}
and 
\begin{align}
	\vec{R} = 	\myvec{2 \\  1 \\ 6 }
\end{align}
		be any point.  Find the image of $\vec{R}$ in the plane $\vec{P}$.
	\item A vector 
\begin{align}
	\vec{a} = 	\myvec{\alpha \\  2 \\ \beta }
\end{align}
lies in the plane of the vectors 
\begin{align}
	\vec{b} = 	\myvec{1\\  1 \\ 0}, 
	\vec{b} = 	\myvec{1\\  -1 \\ 4}
\end{align}
		If $\vec{a}$ bisects the angle between $\vec{b}, \vec{c}$, find a condition on $\vec{a}$.
	\item If the system of linear equations 
\begin{align}
	2x + 2ay + az & = 1
	\\
	2x + 3by + bz & = 1
	\\
	2x + 4cy + cz & = 1
\end{align}
where $a, b, c \in \mathbb{R}$ are nonzero and distinct; has a nonzero solution, find a condition on $a,b,c$.
\item Let 
\begin{align}
	\vec{A} = 	\myvec{1\\   0}, 
	\vec{B} = 	\myvec{6\\ 2 },
	\vec{C} = 	\myvec{\frac{3}{2}\\[1em] 6}
\end{align}
be the vertices of a triangle.  If $\vec{P}$ is a point inside the triangle $ABC$ such that the triangles $APC,APB$ and $BPC$ have equal areas, then find the length of the line segment $PQ$, where  
\begin{align}
	\vec{P} = 	\myvec{-\frac{7}{6}\\[1em] -\frac{1}{3} } 
\end{align}
\end{enumerate}
\section{JNTU}
\renewcommand{\theequation}{\theenumi}
%\begin{enumerate}[label=\arabic*.,ref=\theenumi]
\begin{enumerate}[label=\thesection.\arabic*.,ref=\thesection.\theenumi]
\numberwithin{equation}{enumi}
\item Find the value of $k$ such that the rank of the matrix
\begin{align}
	\myvec{1 & 2 & 3\\2 & k & 7 \\ 3 & 6 & 10} 
\end{align}
is 2.
\item Find the LU decoomposition of 
\begin{align}
	\vec{A} = 	\myvec{1 &  3\\ 4 & -1} 
\end{align}
\item If a square matrix $\vec{A}$ has an eigenvalue $\lambda $, then what is the eigenvalue of the matrix $k\vec{A}$, where $k \ne 0$ is a scalar.
\item If a matrix 
\begin{align}
	\vec{A} = 	\myvec{-1 & 0 & 0\\2 & -3 & 0 \\ 1 & 4 & 2}
\end{align}
then what are the eigenvalues of $\vec{A}^2$?
\item Factorize the matrix 
\begin{align}
		\myvec{2 & -3 & 1\\3 & 4 & 2 \\ 2 & -3 & 4}
\end{align}
by the LU decomposition method.
\item For what values of $\lambda$ and $\mu$ do the system of equations
\begin{align}
	x+y+z &= 6
	\\
	x+2y+3z &= 10
	\\
	x+2y+\lambda z = \mu
\end{align}
have
		\begin{enumerate}
			\item no solution
			\item unique solution
			\item more than one solution
				\end{enumerate}
			\item Find the value of $k$ for which the system of equations
\begin{align}
	\brak{k+1}x+y &= 4k
	\\
	kx+\brak{k-3}y &= 3k-1
\end{align}
has infinitely many solutions.
\item Verify Cayley-Hamilton Theorem for the matrix 
\begin{align}
	\vec{A} = 	\myvec{1 & 2 & 0\\-1 & -1 & 2 \\ 1 & 2 & 1}
\end{align}
		and obtain $\vec{A}^{-1}$ and $\vec{A}^{3}$.
	\item Reduce the quadratic form 
\begin{align}
	3x^2+3y^2+3z^2-2yz+2zx+2xy
\end{align}
to its canonical form.

\end{enumerate}
\end{document}


